\documentclass[sheet=1, prefix]{dexercise}

\title{Computer Security}
\author{Daniel Knaack}

\begin{document}

\task[Memory Addresses]

\begin{enumerate}
  \item

  \item
    \begin{enumerate}
      \item
        The most-significant bits \texttt{vaddr[63:p]} determine the page frame number
        and the page offset is determined by the least-significant bits \texttt{vaddr[p-1:0]}.
      \item
        The bits that are used for the page offset are identical since they are
        used for indexing into the page itself.
    \end{enumerate}
  \item
    For x86\_64, you can usually tell by looking at the most-significant bits
    since only 48 bits are used to get the physical memory address.
    If the most-significant bits are set to $1$, then this indicates that it is a kernel memory address.
    Otherwise, it indicates a user-level address.
  \item
    \begin{enumerate}
      \item
      \item
    \end{enumerate}
  \item
    This makes it easy to compute the offset from the address.
    Computing the offset will only require a masking operation instead of an
    integer division operation for non-power-of-two page sizes.
  \item
    Cache contention attack?
  \item

  \item
    Depending on the specific CPU, the TLB is flushed or if not, then there is an ID for the specific process,
    such that another process cannot simply access a memory address stored for a different process.
\end{enumerate}

\task[Page Fault Side Channel]

\end{document}
